\documentclass{article}
\usepackage{graphicx}
\usepackage{float}
\usepackage{pdfpages}
\usepackage[legalpaper, portrait, margin=1in]{geometry}
\usepackage{enumitem}
\usepackage{minted}
\usepackage{amsmath}

\author{Xining Li, Debashis Kayal}
\title {Ensemble Learning Report}

\begin{document}
\maketitle

\section{Motivation}

The performance of any classifier, or for that matter any machine learning task, depends crucially on the quality of the available data. 


In particular, existing machine learning algorithms are trying to find function within the hypothesis space that can approximate the following formulas. 

\begin{equation}
    f^*(x) = {\mathrm {argmax}}_{Y=y} P (Y=y|X=x)
\end{equation}

The above formula does not address the existence of mislabeled data. Therefore, any classifier's performance will be harmed by noise in dataset labels, which is to be anticipated; the relevant question is by how much? 

In this project, we will be investigating the performance of ensemble learning algorithms with mislabeled data. We will be exploring different datasets, different machine learning algorithms, and different machine learning approaches; we will be juxtaposing training data without mislabeling and with mislabeling under the combinations of these. 

\section{MINST Dataset}

The MNIST database of handwritten digits, available from this page, has a training set of 60,000 examples, and a test set of 10,000 examples. 

\inputminted[firstline=16,lastline=20,frame=single,framesep=10pt]{python}{minst/main.py}

\begin{figure}[H]
    \centering
    \includegraphics[width=\textwidth/2]{report-resources/minst/simple-minst-sample-1.pdf}
    \caption{MINST data handwriting example}
    \label{fig:let1}
\end{figure}

The Figure \ref{fig:let1} is a letter 1 in the MINST dataset. 

\subsection{Without Data Mislabeling}

The following algorithms run on the data without mislabeling. 

\inputminted[firstline=30,lastline=77,frame=single,framesep=10pt]{python}{minst/main.py}

\begin{figure}[H]
    \centering
    \input{report-resources/minst/f1_scores.tex}
    \caption{F1 Scores Table}
\end{figure}


\subsection{With Data Mislabeling}

We use the following data to intentionally create misslabeled data

\inputminted[frame=single,framesep=10pt]{python}{minst/do_miss_label.py}


\inputminted[firstline=93,lastline=142,frame=single,framesep=10pt]{python}{minst/main.py}


\begin{figure}[H]
    \centering
    \input{report-resources/minst/misslabeled_f1_scores.tex}
    \caption{Misslabled data F1 Scores Table}
\end{figure}

\section{The CIFAR-10 dataset}
The CIFAR-10 dataset consists of 60000 32x32 colour images in 10 classes, with 6000 images per class. There are 50000 training images and 10000 test images.

The dataset is divided into five training batches and one test batch, each with 10000 images. The test batch contains exactly 1000 randomly-selected images from each class. The training batches contain the remaining images in random order, but some training batches may contain more images from one class than another. Between them, the training batches contain exactly 5000 images from each class.

Here are the classes in the dataset, as well as 10 random images from each:

\begin{figure}[H]
    \centering
    \includegraphics[width=\textwidth/2]{report-resources/cifar-10-images/show_labels.png}
    \caption{CFAR10 data example}
\end{figure}

CFAR10 is a more complicated dataset, and we will be using the CNN for the feature extraction. 

After the feature extraction, we will be using the following ways for the classification. 
\begin{itemize}
    \item Softmax
    \item Random Forest
\end{itemize}

\subsubsection{Without Data Mislabeling}

\inputminted[frame=single,framesep=10pt]{python}{cifar10/main.py}

\begin{table}[H]
    \centering
    \begin{tabular}{rrrrr}
        CNN &  CNN+RandomForest   \\
        0.7885768705224767 &       0.8059692064910149  \\
    \end{tabular}
\end{table}

\subsubsection{With Data Mislabeling}

We applied 30\% of mislabeled data. 

\inputminted[firstline=32,lastline=54,frame=single,framesep=10pt]{python}{cifar10-mislabeled/main.py}


\begin{table}[H]
    \centering
    \begin{tabular}{rrrrr}
        CNN &  CNN+RandomForest   \\
        0.7371853465743461 &       0.7546625150853542  \\
    \end{tabular}
\end{table}


\end{document}


